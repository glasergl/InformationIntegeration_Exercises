\documentclass{scrartcl}

% neccesarry for style
\usepackage[english]{babel}
\usepackage[utf8]{inputenc}
\usepackage[T1]{fontenc}
\usepackage{lmodern}

% tools for text
\usepackage{csquotes}
\usepackage{url}
\usepackage{hyperref}

% graphics
\usepackage{graphicx}
\usepackage{xcolor}
\usepackage{tikz}

% maths
\usepackage{amsmath,amssymb,amstext,amsthm}

% programming
\usepackage{listings}

\usepackage{wasysym} %lightning

\input{required/settings.tex}
\input{required/commands.tex}

\renewcommand{\labelenumi}{\alph{enumi})}
\begin{document}
	\begin{center}
		\LARGE
		Information Integration -- Exercise 3 -- Gabriel Glaser
	\end{center}
	
	\section*{Task 1: Mediator-Wrapper Architecture}
	\begin{enumerate}
		\item
		
		\item
		
		\item
		
		\item
		
		\item
		
		\item
		
	\end{enumerate}
	
	\section*{Task 2: Stable Marriage Algorithm}
	\begin{enumerate}
		\item\phantom{phantom}
		\begin{center}
			\begin{tabular}{c|c}
				Men & Women\\
				\hline
				A: 1, 2 & 1: A, B\\
				B: 1, 2 & 2: A. B
			\end{tabular}
		\end{center}
		\begin{itemize}
			\item (A,1) \& (B,2) is stable, because A doesn't prefer 2 and 1 doesn't prefer B.
			\item Similarly (B,1) \& (A,2) is stable, because B doesn't prefer 2 and 1 doesn't prefer A.
		\end{itemize}
		
		\item The algorithm needs exactly as many steps as men are there.
		This is because, during the algorithm each man will propose to a woman who didn't have an offer before.
		Thus, she accepts and also, won't get further proposals which may could lead to a change (break up and new search for the man who lost the woman who he initially proposed to).
		
		\item Let's assume each man $m_i$ has the priority list $(w_1, w_2, ..., w_n)$.
		It is the case that each man $m_i$ performs at least 1, but at most $n$ proposals.
		Also, it is not possible that any two men $m_i, m_j, i\neq j$ have the same amount of proposals, because then both of their last proposal would address the same woman (which is not allowed).
		The only way of distributing the (pair-wise different) number of proposals of $n$ men would be that
		\begin{itemize}
			\item one man has 1 proposal
			\item one man has 2 proposals
			\item one man has 3 proposals
			\item ...
			\item one man has $n$ proposals
		\end{itemize}
		Therefore, there are
		\[\sum_{i=1}^ni=\frac{n(n+1)}{2}\]
		proposals.
		
		\item\phantom{phantom}
		\begin{center}
			\begin{tabular}{|l|}
				\hline
				Adam $\to$ Beth $\surd$\\\hline
				Bill $\to$ Diane $\surd$\\\hline
				Carl $\to$ Beth $\surd$ $\quad$ Adam loses partner\\\hline
				Adam $\to$ Amy $\surd$\\\hline
				Dan $\to$ Amy \lightning\\\hline
				Dan $\to$ Diane $\surd$ $\quad$ Bill loses partner\\\hline
				Bill $\to$ Beth \lightning\\\hline
				Bill $\to$ Amy \lightning\\\hline
				Bill $\to$ Cara $\surd$\\\hline
				Eric $\to$ Beth \lightning\\\hline
				Eric $\to$ Diane $\surd$ $\quad$ Dan loses partner\\\hline
				Dan $\to$ Cara \lightning\\\hline
				Dan $\to$ Beth \lightning\\\hline
				Dan $\to$ Ellen $\surd$\\
				\hline
			\end{tabular}
		\end{center}
		Result: (Adam, Amy), (Bill, Cara), (Carl, Beth), (Dan, Ellen), (Eric, Diane).
		
		\item\phantom{phantom}
		\begin{center}
			\begin{tabular}{|l|}
				\hline
				Amy $\to$ Eric $\surd$\\\hline
				Beth $\to$ Carl $\surd$\\\hline
				Cara $\to$ Bill $\surd$\\\hline
				Diane $\to$ Adam $\surd$\\\hline
				Ellen $\to$ Dan $\surd$\\
				\hline
			\end{tabular}
		\end{center}
	Result: (Amy, Eric), (Beth, Carl), (Cara, Bill), (Diane, Adam), (Ellen, Dan).
		
	\end{enumerate}
	
	\section*{Task 3: Schema Matching}
	\begin{enumerate}
		\item
		
		\item
		
		\item
		
		\item
		
		\item
		
		\item
		
	\end{enumerate}
\end{document}
