\documentclass{scrartcl}

% neccesarry for style
\usepackage[english]{babel}
\usepackage[utf8]{inputenc}
\usepackage[T1]{fontenc}
\usepackage{lmodern}

% tools for text
\usepackage{csquotes}
\usepackage{url}
\usepackage{hyperref}

% graphics
\usepackage{graphicx}
\usepackage{xcolor}
\usepackage{tikz}

% maths
\usepackage{amsmath,amssymb,amstext,amsthm}

% programming
\usepackage{listings}

\usepackage{wasysym} %lightning

\input{required/settings.tex}
\input{required/commands.tex}

\renewcommand{\labelenumi}{\alph{enumi})}
\begin{document}
	\begin{center}
		\LARGE
		Information Integration -- Exercise 3 -- Gabriel Glaser
	\end{center}
	
	\section*{Task 1: Mediator-Wrapper Architecture}
	\begin{enumerate}
		\item
		
		\item
		
		\item
		
		\item
		
		\item
		
		\item
		
	\end{enumerate}
	
	\section*{Task 2: Stable Marriage Algorithm}
	\begin{enumerate}
		\item\phantom{phantom}
		\begin{center}
			\begin{tabular}{c|c}
				Men & Women\\
				\hline
				A: 1, 2 & 1: A, B\\
				B: 1, 2 & 2: A. B
			\end{tabular}
		\end{center}
		\begin{itemize}
			\item (A,1) \& (B,2) is stable, because A doesn't prefer 2 and 1 doesn't prefer B.
			\item Similarly (B,1) \& (A,2) is stable, because B doesn't prefer 2 and 1 doesn't prefer A.
		\end{itemize}
		
		\item The algorithm needs exaclty as many steps as men are there.
		This is because, during the algorithm each man will propose to a woman who didn't have an offer before.
		Thus, she accepts and also, won't get further proposals which may could lead to a change (break up and new search for the man who lost the woman who he initially proposed to).
		
		\item
		
		\item
		
		\item
		
	\end{enumerate}
	
	\section*{Task 3: Schema Matching}
	\begin{enumerate}
		\item\phantom{phantom}
		\begin{itemize}
			\item \textbf{D1}: Columns \textit{SSN} and \textit{Birthday}
			\item \textbf{D2}: Last two columns
		\end{itemize}
		
		\item\phantom{phantom}
		\begin{itemize}
			\item\textbf{D1}:
			\begin{center}
				\fbox{
					\begin{tabular}{l}
						\textcolor{gray}{\# SQL code to create a restricted view}\\
						\textbf{CREATE VIEW} WrapperTableD1 \textbf{AS}\\
						\textbf{SELECT} SSN, Birthday \textbf{FROM} TableD1;\\
						\textbf{ALTER} \textbf{TABLE} WrapperTableD1\\
						\textbf{RENAME} \textbf{COLUMN} Birthday \textbf{TO} BirthYear;
					\end{tabular}
				}\\
				\vspace{0.8cm}
				\fbox{
					\begin{tabular}{l}
						\textcolor{gray}{\# Python}\\
						\textbf{CREATE VIEW} WrapperTableD1 \textbf{AS}\\
						\textbf{SELECT} SSN, Birthday \textbf{FROM} TableD1;\\
						\textbf{ALTER} \textbf{TABLE} WrapperTableD1\\
						\textbf{RENAME} \textbf{COLUMN} Birthday \textbf{TO} BirthYear;
					\end{tabular}
				}
			\end{center}
			
			\item\textbf{D2}:
			\begin{center}
				\fbox{
					\begin{tabular}{l}
						\textcolor{gray}{\# SQL code to create table for relevant data}\\
						\textbf{CREATE TABLE} WrapperTableD2 (\\
						\hspace{0.5cm}AverageSalary \textbf{INT}(255),\\
						\hspace{0.5cm}SSN \textbf{CHAR}(12)\\
						);
					\end{tabular}
				}\\
				\vspace{0.8cm}
				\fbox{
					\begin{tabular}{l}
						\textcolor{gray}{\# python-like script to extract information and fill in table}\\
						\textbf{for} row \textbf{in} csv\_rows:\\
						\hspace{0.5cm}cells = row.split(",")\\
						\hspace{0.5cm}salary = \textbf{int}(remove\_last\_char(cells[4]))\\
						\hspace{0.5cm}ssn = cells[5]\\
						\hspace{0.5cm}run\_sql\_query(\\
						\hspace{1cm}\textbf{INSERT INTO} WrapperTableD2\\
						\hspace{1.5cm}\textbf{VALUES}(ssn, salary)\\
						\hspace{1cm});\\
						\hspace{0.5cm})
					\end{tabular}
				}
			\end{center}
		\end{itemize}
		
		\item Mediator
		\begin{center}
			\fbox{
				\begin{tabular}{l}
					\textbf{CREATE} \textbf{VIEW} tmp \textbf{AS}\\
					\textbf{SELECT} * \textbf{FROM} WrapperTableD1, WrapperTableD2 \textbf{WHERE}\\
					WrapperTableD1.SSN = WrapperTableD2.SSN;\\
					\textbf{CREATE} \textbf{VIEW} IntegratedTable \textbf{AS}\\
					\textbf{SELECT} BirthYear, AverageSalary \textbf{FROM} tmp;
				\end{tabular}
			}
		\end{center}
		
		\item
		
		\item
		
		\item
		
	\end{enumerate}
\end{document}
