\documentclass{scrartcl}

% neccesarry for style
\usepackage[english]{babel}
\usepackage[utf8]{inputenc}
\usepackage[T1]{fontenc}
\usepackage{lmodern}

% tools for text
\usepackage{csquotes}
\usepackage{url}
\usepackage{hyperref}

% graphics
\usepackage{graphicx}
\usepackage{xcolor}
\usepackage{tikz}

% maths
\usepackage{amsmath,amssymb,amstext,amsthm}

% programming
\usepackage{listings}

\usepackage{wasysym} %lightning

\input{required/settings.tex}
\input{required/commands.tex}

\renewcommand{\labelenumi}{\alph{enumi})}
\begin{document}
	\begin{center}
		\LARGE
		Information Integration -- Exercise 2 -- Gabriel Glaser
	\end{center}
	\vspace{1cm}
	\section*{Task 1: Heterogeneity}
	\begin{enumerate}
		\item 
		\begin{itemize}
			\item \textit{Syntactic} (Hardware, Software, Interface): No, only one source.
			
			\item \textit{Structural} (Data model, schematic): No, only one source
			
			\item \textit{Semantic}:
			\begin{itemize}
				\item Name conflict: synonym labels \enquote{year} and \enquote{YEAR}.
				\item Identity: Some of the first results likely refer to the same album, but couldn't be differentiated, because different or incomplete representations.
				\item Value conflict: Track titles: \textit{You know you're right} vs. \textit{You Know You're Right} among first Nirvana results.
			\end{itemize}
			
		\end{itemize}
	
		\item Extensions to scenario to satisfy missing heterogeneity types:
		\begin{itemize}
			\item \textit{Syntactic}: Consider a CD data source from America which likely has a longer response time. (different hardware)
			
			\item \textit{Structural}: Consider a CD data source which returns an XML file. (different data model)
		\end{itemize}
	\end{enumerate}
	
	\section*{Task 3: Distributed DBMS}
	\begin{enumerate}
		\item Distribution:
		\begin{itemize}
			\item \textit{Physical}: Servers are located in different physical/geographical locations.
			Therefore, they don't share hardware components (except network).
			
			\item \textit{Logical}: Result of application requirements, e.g., store data on different servers to handle network failure or implement caching for more speed.
		\end{itemize}
	
		\item Contained autonomy types:
		\begin{itemize}
			\item \textit{Design}: Choose to store DB in 3NF ($3^{\text{rd}}$ normal form) or as a \enquote{BigTable} with less normalization.
			\item \textit{Communication}: Decision to communicate with SQL and web form (\textit{decision on specific query languages}).
			Also, they decide to allow write access to older table but not on newer table (\textit{decision on what query capabilities to support}).
		\end{itemize}
	
		\item Execution autonomy is the last type of autonomy.
		For instance, a system could block queries from some countries.
		
		\item Syntactic heterogeneity:
		\begin{itemize}
			\item \textit{Hardware}: This type of heterogeneity can be noticed when some datasources process a query faster than others (different CPU/bandwidth).
			
			\item \textit{Software}: Different datasources could be stored using different operation systems, e.g., need to use different file separators (\enquote{/} for Linux, \enquote{\textbackslash} for Windows).
			
			\item \textit{Interface}: For instance, access to datasource, i.e., pass a query, via URL containing various parameters vs. access via SQL query given to an URL as string.
		\end{itemize}
	
		\item
		\begin{enumerate}
			\item Value vs. relation heterogeneity:
			\begin{center}
				\fbox{
				\begin{tabular}{l}
					Employee(\underline{ID}, Firstname, Lastname, isManager, department, gender)
				\end{tabular}
				}\\
				\vspace{0.3cm}
				vs.\\
				\vspace{0.3cm}
				\fbox{
				\begin{tabular}{l}
					MaleEmployee(\underline{ID}, Firstname, Lastname, isManager, department)\\
					FemaleEmployee(\underline{ID}, Firstname, Lastname, isManager, department)
				\end{tabular}
				}
			\end{center}
		
			\item Value vs. attribute heterogeneity:
			\begin{center}
				\fbox{
					\begin{tabular}{l}
						Employee(\underline{ID}, Firstname, Lastname, isManager, department, gender)
					\end{tabular}
				}\\
				\vspace{0.3cm}
				vs.\\
				\vspace{0.3cm}
				\fbox{
					\begin{tabular}{l}
						Employee(\underline{ID}, Firstname, Lastname, isManager, department, isFemale, isMale)
					\end{tabular}
				}
			\end{center}
			
			\item Different labels of attributes (in comparison with the original):
			\begin{center}
				\fbox{
					\begin{tabular}{l}
						Employee(\underline{ID}, Firstname, Lastname, isManager, \textit{workArea})
					\end{tabular}
				}
			\end{center}
			
			\item Normalized vs non-normalized schema (original is normalized):
			\begin{center}
				\fbox{
					\begin{tabular}{l}
						Employee(\underline{ID}, Firstname, Lastname, department)\\
						Managers(\underline{MID}, \underline{ID $\to$ Employee})
					\end{tabular}
				}
			\end{center}
		\end{enumerate}
	\end{enumerate}
	
	\section*{Task 4: Integrating publication data}
	\begin{enumerate}
		\item
		\begin{itemize}
			\item \textbf{Syntactic}:
			Likely a hardware heterogeneity, because Amazon is a much bigger cooperation than ACM and DBLP.
			For instance, Amazon is likely more reliable.
			\item \textbf{Structural}:
			Different data model between ACM and DBLP (XML vs. BibTex).
			\item \textbf{Semantic}:
			Value conflict of attribute \textit{publisher} between ACM and DBLP results (\enquote{ideaGroup} vs. \enquote{ACM}).
		\end{itemize}
		\newpage
		\item \textbf{DBLP}:
		\begin{center}
			\fbox{
				\begin{tabular}{l}
					Publications(\underline{P-ID}, \emph{publicationType}, \emph{key}, \emph{mdate}, title, publisher, year, isbn, url)\\
					Editors(\underline{E-ID}, name, \underline{P-ID $\to$ Publications})\\
					%DBLP(\underline{DBLP-ID}, \underline{P-ID $\to$ Publications})
				\end{tabular}
			}
		\end{center}
		
		\textbf{ACM}:
		\begin{center}
			\fbox{
				\begin{tabular}{l}
					Publications(\underline{P-ID}, \emph{publicationType}, label, author, title, journal, issue-date, volume, \\
					\hspace{3cm} number, month, year, issn, pages, numpages, url, doi, acmid,\\
					\hspace{3cm} publisher, address)\\
					%ACM(\underline{ACM-ID},\underline{P-ID $\to$ Publications})
				\end{tabular}
			}
		\end{center}
		
		\textbf{Amazon}:
		\begin{center}
			\fbox{
				\begin{tabular}{l}
					Items(\underline{Item-ID}, name, \underline{Prize-ID $\to$ Prizes}, description, isbn-10, isbn-13, publisher,\\
					\hspace{3cm} publishDate, language, size, weight, numberOfPages, review,\\
					\hspace{3cm} itemType)\\
					Authors(\underline{Author-ID}, firstname, lastname, \underline{Item-ID $\to$ Items})\\
					Prizes(\underline{Prize-ID}, \underline{Item-ID $\to$ Items}, description [\textit{e.g., new or used}], prizeInDollar)\\
					%Amazon(\underline{A-ID}, \underline{Item-ID $\to$ Items})
				\end{tabular}
			}
		\end{center}
		
		\item Attribute matches from target schema attributes to matched attributes:
		\begin{itemize}
			\item \textbf{DBLP}: Match title $\to$ title, publisher $\to$ publisher, pubType $\to$ publicationType, pubDate $\to$ mdate, editon $\to$ \textit{not given}\\
			
			Extract authors from \textit{Editor}-table, then match
			firstname $\to$ firstname, lastname $\to$ lastname, but authorType $\to$ \textit{not given}.
			
			\item \textbf{ACM}: Match title $\to$ title, publisher $\to$ publisher, pubType $\to$ publicationType, pubDate $\to$ \textit{only month, year given}, editon $\to$ volume\\
			
			Extract single names by splitting on appropriately, then match
			firstname $\to$ \textit{only first letter given}, lastname $\to$ lastname, authorType $\to$ \textit{not given}.
			
			\item \textbf{Amazon}: Match title $\to$ name, publisher $\to$ publisher, pubType $\to$ itemType, pubDate $\to$ publishDate, editon $\to$ \textit{not given}\\
			
			Extract authors from \textit{Authors}-table of Amazon, then match
			firstname $\to$ firstname, lastname $\to$ \textit{lastname}, authorType $\to$ \textit{not given}.
		\end{itemize}
		
	\end{enumerate}
	
\end{document}
